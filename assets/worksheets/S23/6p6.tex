\documentclass[11pt]{amsart}
%prepared in AMSLaTeX, under LaTeX2e
\addtolength{\oddsidemargin}{-.9in} 
\addtolength{\evensidemargin}{-.9in}
\addtolength{\topmargin}{-.9in}
\addtolength{\textwidth}{1.5in}
\addtolength{\textheight}{1.5in}

\renewcommand{\baselinestretch}{1.05}

\usepackage{verbatim} % for "comment" environment

\usepackage{palatino}

\usepackage[final]{graphicx}

\usepackage{tikz}
\usetikzlibrary{positioning}

\usepackage{enumitem,xspace,fancyvrb}

\newtheorem*{thm}{Theorem}
\newtheorem*{defn}{Definition}
\newtheorem*{example}{Example}
\newtheorem*{problem}{Problem}
\newtheorem*{remark}{Remark}

\DefineVerbatimEnvironment{mVerb}{Verbatim}{numbersep=2mm,frame=lines,framerule=0.1mm,framesep=2mm,xleftmargin=4mm,fontsize=\footnotesize}

% macros
\usepackage{amssymb}
\newcommand{\bA}{\mathbf{A}}
\newcommand{\bB}{\mathbf{B}}
\newcommand{\bE}{\mathbf{E}}
\newcommand{\bF}{\mathbf{F}}
\newcommand{\bJ}{\mathbf{J}}

\newcommand{\bb}{\mathbf{b}}
\newcommand{\bi}{\mathbf{i}}
\newcommand{\bj}{\mathbf{j}}
\newcommand{\bk}{\mathbf{k}}
\newcommand{\br}{\mathbf{r}}
\newcommand{\bu}{\mathbf{u}}
\newcommand{\bv}{\mathbf{v}}
\newcommand{\bw}{\mathbf{w}}
\newcommand{\bx}{\mathbf{x}}

\newcommand{\ppr}[1]{\frac{\partial #1}{\partial r}}
\newcommand{\ppt}[1]{\frac{\partial #1}{\partial t}}
\newcommand{\ppx}[1]{\frac{\partial #1}{\partial x}}
\newcommand{\ppy}[1]{\frac{\partial #1}{\partial y}}
\newcommand{\ppz}[1]{\frac{\partial #1}{\partial z}}

\newcommand{\Div}{\ensuremath{\nabla\cdot}}
\newcommand{\Curl}{\ensuremath{\nabla\times}}

\newcommand{\eps}{\epsilon}
\newcommand{\grad}{\nabla}
\newcommand{\ip}[2]{\ensuremath{\left<#1,#2\right>}}
\newcommand{\lam}{\lambda}
\newcommand{\lap}{\triangle}

\newcommand{\RR}{\mathbb{R}}
\newcommand{\ZZ}{\mathbb{Z}}
\newcommand{\prob}[1]{\bigskip\noindent\textbf{#1.}\quad }

\newcommand{\Matlab}{\textsc{Matlab}\xspace}

\newcommand{\ds}{\displaystyle}

\begin{document}
\scriptsize \noindent Math 253 Calculus III (Bueler) \hfill 27 April 2023 \fbox{\emph{Not turned in!}}
\normalsize\medskip

\Large\centerline{\textbf{Worksheet: Surface integrals}}
\medskip
\normalsize

\thispagestyle{empty}

\bigskip

\prob{1}  Sketch the parameterized surface $S$ given by $\br(u,v)=\left<\cos v,\sin v,u\right>$ for $0 \le u \le 5$ and $0 \le v \le \pi$.  Then compute the surface integral
    $$\iint_S z \,dS = \hspace{5.0in}$$
% = 25 \pi
\vfill

\prob{2}  Let $S$ be the part of the graph (surface) $z=1-x^2-y^2$ for which $z \ge 0$.  Parameterize this surface.  Set-up, and compute, a surface integral for its area.
% \br(u,v)=\left<u\cos v,u\sin v,1-u^2\right>
% \iint_S 1 \,dS = (pi/6) (5^(3/2) - 1)
\vfill

%\noindent \hrule
%
%\bigskip
%\centerline{\footnotesize \textsc{extra space}}
%\vfill
\end{document}
