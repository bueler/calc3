\documentclass[11pt]{amsart}
%prepared in AMSLaTeX, under LaTeX2e
\addtolength{\oddsidemargin}{-.9in} 
\addtolength{\evensidemargin}{-.9in}
\addtolength{\topmargin}{-.9in}
\addtolength{\textwidth}{1.5in}
\addtolength{\textheight}{1.5in}

\renewcommand{\baselinestretch}{1.05}

\usepackage{verbatim} % for "comment" environment

\usepackage{palatino}

\usepackage[final]{graphicx}

\usepackage{tikz}
\usetikzlibrary{positioning}

\usepackage{enumitem,xspace,fancyvrb}

\newtheorem*{thm}{Theorem}
\newtheorem*{defn}{Definition}
\newtheorem*{example}{Example}
\newtheorem*{problem}{Problem}
\newtheorem*{remark}{Remark}

\DefineVerbatimEnvironment{mVerb}{Verbatim}{numbersep=2mm,frame=lines,framerule=0.1mm,framesep=2mm,xleftmargin=4mm,fontsize=\footnotesize}

% macros
\usepackage{amssymb}
\newcommand{\bA}{\mathbf{A}}
\newcommand{\bB}{\mathbf{B}}
\newcommand{\bE}{\mathbf{E}}
\newcommand{\bF}{\mathbf{F}}
\newcommand{\bJ}{\mathbf{J}}

\newcommand{\bb}{\mathbf{b}}
\newcommand{\bi}{\mathbf{i}}
\newcommand{\bj}{\mathbf{j}}
\newcommand{\bk}{\mathbf{k}}
\newcommand{\br}{\mathbf{r}}
\newcommand{\bu}{\mathbf{u}}
\newcommand{\bv}{\mathbf{v}}
\newcommand{\bw}{\mathbf{w}}
\newcommand{\bx}{\mathbf{x}}

\newcommand{\ppr}[1]{\frac{\partial #1}{\partial r}}
\newcommand{\ppt}[1]{\frac{\partial #1}{\partial t}}
\newcommand{\ppx}[1]{\frac{\partial #1}{\partial x}}
\newcommand{\ppy}[1]{\frac{\partial #1}{\partial y}}
\newcommand{\ppz}[1]{\frac{\partial #1}{\partial z}}

\newcommand{\Div}{\ensuremath{\nabla\cdot}}
\newcommand{\Curl}{\ensuremath{\nabla\times}}

\newcommand{\eps}{\epsilon}
\newcommand{\grad}{\nabla}
\newcommand{\ip}[2]{\ensuremath{\left<#1,#2\right>}}
\newcommand{\lam}{\lambda}
\newcommand{\lap}{\triangle}

\newcommand{\RR}{\mathbb{R}}
\newcommand{\ZZ}{\mathbb{Z}}
\newcommand{\prob}[1]{\bigskip\noindent\textbf{#1.}\quad }

\newcommand{\Matlab}{\textsc{Matlab}\xspace}

\newcommand{\ds}{\displaystyle}

\begin{document}
\scriptsize \noindent Math 253 Calculus III (Bueler) \hfill 17 April 2023 \fbox{\emph{Not turned in!}}
\normalsize\medskip

\Large\centerline{\textbf{Worksheet: Computing potentials}}
\medskip
\normalsize

\thispagestyle{empty}

\bigskip

\prob{1}  Is the 2D vector field $\bF=\left<\sin y, x\cos y\right>$ conservative?  If it is, compute a potential $f(x,y)$ so that $\bF = \grad f$.
% yes; P_y = \cos y, Q_x = \cos y;  f(x,y) = x \sin y
\vfill

\prob{2}  Is the 3D vector field $\bF=\left<2x,e^z,ye^z-1\right>$ conservative?  If it is, compute a potential $f(x,y,z)$ so that $\bF = \grad f$.
% yes: curl F = <e^z-e^z,-(0-0),0-0> = 0; f(x,y,z) = x^2 + y e^z - z
\vfill

\prob{3}  Is the 3D vector field $\bF=\left<y \sin(x),e^z,y\right>$ conservative?  If it is, compute a potential $f(x,y,z)$ so that $\bF = \grad f$.
% no:  curl F = (e^z - 1) ihat - ... != 0
\vfill

\clearpage\newpage
\prob{4}  Is the 2D vector field $\bF=(2xye^{x^2y}) \bi + (x^2 e^{x^2 y}) \bj$ conservative?  If it is, compute a potential $f(x,y)$ so that $\bF = \grad f$.
% #125
% yes;  P_x = 2x e^.. + 2xy e^.. x^2; Q_x = 2x e^.. + x^2 e^.. 2xy;  f(x,y) = e^{x^2 y}
\vfill

\prob{5}  Is the 3D vector field $\bF=(2xy) \bi + (x^2 + 2 y z) \bj + y^2 \bk$ conservative?  If it is, compute a potential $f(x,y,z)$ so that $\bF = \grad f$.
% #123
% yes; f(x,y,z) = x^2 y + y^2 z
\vfill

\noindent \hrule

\bigskip
\centerline{\footnotesize \textsc{extra space}}
\vfill
\end{document}
