\documentclass[11pt]{amsart}
%prepared in AMSLaTeX, under LaTeX2e
\addtolength{\oddsidemargin}{-.9in} 
\addtolength{\evensidemargin}{-.9in}
\addtolength{\topmargin}{-.9in}
\addtolength{\textwidth}{1.5in}
\addtolength{\textheight}{1.5in}

\renewcommand{\baselinestretch}{1.05}

\usepackage{verbatim} % for "comment" environment

\usepackage{palatino}

\usepackage[final]{graphicx}

\usepackage{tikz}
\usetikzlibrary{positioning}

\usepackage{enumitem,xspace,fancyvrb}

\newtheorem*{thm}{Theorem}
\newtheorem*{defn}{Definition}
\newtheorem*{example}{Example}
\newtheorem*{problem}{Problem}
\newtheorem*{remark}{Remark}

\DefineVerbatimEnvironment{mVerb}{Verbatim}{numbersep=2mm,frame=lines,framerule=0.1mm,framesep=2mm,xleftmargin=4mm,fontsize=\footnotesize}

% macros
\usepackage{amssymb}
\newcommand{\bA}{\mathbf{A}}
\newcommand{\bB}{\mathbf{B}}
\newcommand{\bE}{\mathbf{E}}
\newcommand{\bF}{\mathbf{F}}
\newcommand{\bJ}{\mathbf{J}}

\newcommand{\bb}{\mathbf{b}}
\newcommand{\bi}{\mathbf{i}}
\newcommand{\bj}{\mathbf{j}}
\newcommand{\bk}{\mathbf{k}}
\newcommand{\br}{\mathbf{r}}
\newcommand{\bu}{\mathbf{u}}
\newcommand{\bv}{\mathbf{v}}
\newcommand{\bw}{\mathbf{w}}
\newcommand{\bx}{\mathbf{x}}

\newcommand{\ppr}[1]{\frac{\partial #1}{\partial r}}
\newcommand{\ppt}[1]{\frac{\partial #1}{\partial t}}
\newcommand{\ppx}[1]{\frac{\partial #1}{\partial x}}
\newcommand{\ppy}[1]{\frac{\partial #1}{\partial y}}
\newcommand{\ppz}[1]{\frac{\partial #1}{\partial z}}

\newcommand{\Div}{\ensuremath{\nabla\cdot}}
\newcommand{\Curl}{\ensuremath{\nabla\times}}

\newcommand{\eps}{\epsilon}
\newcommand{\grad}{\nabla}
\newcommand{\ip}[2]{\ensuremath{\left<#1,#2\right>}}
\newcommand{\lam}{\lambda}
\newcommand{\lap}{\triangle}

\newcommand{\RR}{\mathbb{R}}
\newcommand{\ZZ}{\mathbb{Z}}
\newcommand{\prob}[1]{\bigskip\noindent\textbf{#1.}\quad }

\newcommand{\Matlab}{\textsc{Matlab}\xspace}

\newcommand{\ds}{\displaystyle}

\begin{document}
\scriptsize \noindent Math 253 Calculus III (Bueler) \hfill 10 March 2023 \fbox{\emph{Not turned in!}}
\normalsize\medskip

\Large\centerline{\textbf{Worksheet: Double integrals over rectangles}}
\medskip
\normalsize

\thispagestyle{empty}

\bigskip

\prob{1}  Evaluate the double (iterated) integral:
$$\int_0^\pi \int_0^{\pi/2} \cos x \sin(3y)\,dx\,dy = \hspace{4.0in}$$
\vfill

\prob{2}  Evaluate the double (iterated) integral:
$$\int_1^2 \int_3^4 x^5 + y^6\,dy\,dx = \hspace{4.5in}$$
\vfill

\prob{3}  Write the integral as an iterated integral in the two different ways:
$$\iint_R e^{\cos(xy)} x^2\,dA \qquad \text{ where } R=[0,1]\times[-1,1] \hspace{3.0in}$$
\vfill

\clearpage\newpage
\prob{4}  Apply the midpoint rule to estimate the integral.  Use $m=2$ points in the $x$-direction and $n=2$ points in the $y$-direction
$$\iint_R \frac{1}{xy}\,dA \qquad \text{ where } R=[1,2]\times[1,3] \hspace{3.0in}$$
\vfill

\prob{5}  Compute the integral in problem \textbf{4} exactly.  How close is the midpoint rule estimate?
\vspace{3.5in}

\end{document}
