\documentclass[12pt]{article}

% Layout.
\usepackage[top=1.2in, bottom=0.75in, left=1in, right=1in, headheight=1.0in, headsep=0pt]{geometry}

% Fonts.
\usepackage{mathptmx}
\usepackage[scaled=0.86]{helvet}
\renewcommand{\emph}[1]{\textsf{\textbf{#1}}}

% TiKZ.
\usepackage{tikz, pgfplots}
\usetikzlibrary{calc}
\pgfplotsset{my style/.append style={axis x line=middle, axis y line=middle, xlabel={$x$}, ylabel={$y$}}}
\pgfplotsset{compat=1.16}

% Misc packages.
\usepackage{amsmath,amssymb,latexsym,bm,array,multicol,enumitem}
\usepackage{graphicx}
\usepackage{xcolor}

% Commands to set various header/footer components.
\makeatletter
\def\doctitle#1{\gdef\@doctitle{#1}}
\doctitle{Use {\tt\textbackslash doctitle\{MY LABEL\}}.}
\def\docdate#1{\gdef\@docdate{#1}}
\docdate{Use {\tt\textbackslash docdate\{MY DATE\}}.}
\def\doccourse#1{\gdef\@doccourse{#1}}
\let\@doccourse\@empty
\def\docscoring#1{\gdef\@docscoring{#1}}
\let\@docscoring\@empty
\def\docversion#1{\gdef\@docversion{#1}}
\let\@docversion\@empty
\makeatother

% Headers and footers layout.
\makeatletter
\usepackage{fancyhdr}
\pagestyle{fancy}
\fancyhf{} % Clears all headers/footers.
\lhead{\emph{\@doctitle\hfill\@docdate} \medskip
\ifnum \value{page} > 1\relax\else\\
\emph{Name: \rule{3.5in}{1pt}\hfill \@docscoring}
\fi}

\rfoot{\emph{\@docversion}}
\lfoot{\emph{\@doccourse}}
\cfoot{\emph{\thepage}}
\renewcommand{\headrulewidth}{0pt}%
\makeatother

% Paragraph spacing
\parindent 0pt
\parskip 6pt plus 1pt

% A problem is a section-like command. Use \problem{5} for a problem worth 5 points.
\newcounter{probcount}
\newcounter{subprobcount}
\setcounter{probcount}{0}

\newcommand{\problem}[1]{%
\par
\addvspace{4pt}%
\setcounter{subprobcount}{0}%
\stepcounter{probcount}%
\makebox[0pt][r]{\emph{\arabic{probcount}.}\hskip1ex}\emph{[#1 points]}\hskip1ex}

\newcommand{\thesubproblem}{\emph{\alph{subprobcount}.}}

% Subproblems are an enumerate-like environment with a consistent
% numbering scheme.  Use \begin{subproblems}\item...\item...\end{subproblems}
\newenvironment{subproblems}{%
\begin{enumerate}[itemindent=8pt,leftmargin=0pt]%
\setcounter{enumi}{\value{subprobcount}}%
\renewcommand{\labelenumi}{\emph{\textsl{\alph{enumi})}} \,}%
}%
{%
\setcounter{subprobcount}{\value{enumi}}%
\end{enumerate}%
}

% Blanks for answers in normal and math mode.
\newcommand{\blank}[1]{\rule{#1}{0.75pt}}
\newcommand{\mblank}[1]{\underline{\hspace{#1}}}
\def\emptybox(#1,#2){\framebox{\parbox[c][#2]{#1}{\rule{0pt}{0pt}}}}

% Misc.
\renewcommand{\d}{\displaystyle}
\newcommand{\ds}{\displaystyle}
\newcommand{\threeopts}{{\small \hspace{-6mm} $\begin{matrix} \text{\textsc{converges}} \\ \text{\textsc{absolutely}} \end{matrix}$ \qquad\qquad $\begin{matrix} \text{\textsc{converges}} \\ \text{\textsc{conditionally}} \end{matrix}$ \qquad\qquad \textsc{diverges}} \bigskip}


\newcommand{\ba}{\mathbf{a}}
\newcommand{\bb}{\mathbf{b}}
\newcommand{\bc}{\mathbf{c}}
\newcommand{\bi}{\mathbf{i}}
\newcommand{\bj}{\mathbf{j}}
\newcommand{\bk}{\mathbf{k}}
\newcommand{\bn}{\mathbf{n}}
\newcommand{\br}{\mathbf{r}}
\newcommand{\bu}{\mathbf{u}}
\newcommand{\bv}{\mathbf{v}}
\newcommand{\bw}{\mathbf{w}}

\newcommand{\bT}{\mathbf{T}}

\newcommand{\grad}{\nabla}
\newcommand{\Div}{\nabla\cdot}

\newcommand{\ip}[2]{\mathrm{\left<#1,#2\right>}}
\newcommand{\vv}[2]{\mathrm{\left<#1,#2\right>}}
\newcommand{\vvv}[3]{\mathrm{\left<#1,#2,#3\right>}}


\doctitle{Math 253: Quiz 6}
\docdate{9 March, 2023}
\doccourse{}
\docversion{}
\docscoring{\fbox{{\LARGE \strut}\hspace{0.8in} / 25}}

\begin{document}
30 minutes maximum.  No aids (book, calculator, etc.) are permitted.  Show all work and use proper notation for full credit.  Answers should be in reasonably-simplified form.  25 points possible.

% like 4.6 # 264
\problem{7}  Consider the function \, {\large $f(x,y) = e^x \cos y$}.

\begin{subproblems}
\item Compute the gradient \, {\large $\grad f(x,y)$}.
\vspace{2.5in}

\item Compute the directional derivative of \, {\large $f$} \, at the point \, {\large $P\left(1,\frac{\pi}{2}\right)$} \, in the direction \, {\large $\bv = - \bi$}.
\vfill
\end{subproblems}


\clearpage\newpage
% like #299
\problem{5}  Find the maximum rate of change of \, {\large $f(x,y)=x \ln y$} \, at the point $(2,1)$, and the direction in which it occurs.
\vfill

% like #289
\problem{5}  Sketch the level curve of \, {\large $f(x,y) = 3 x^2 + 3 y^2$} \, which passes through the point \, {\large $P(1,1)$},\, and draw the gradient vector at \, {\large $P$}.
\vfill


\clearpage\newpage
% like 4.7 #324
\problem{8}  Consider the function \, {\large $f(x,y)=x^3+y^3 - 3 x - 12 y - 2$}.

\begin{subproblems}
\item Find all the critical points.
\vspace{3.0in}

\item For each critical point, use the second derivative test to determine if it is a local minimum, local maximum, or saddle point.
\vfill
\end{subproblems}


\clearpage\newpage
\noindent \emph{Extra Credit. [1 point]} \, Show that the gradient of a function $f(x,y)$ is orthogonal to its level curves.  (\emph{Hint.}  Write down the equation for a level curve.  Suppose the level curve is parameterized.  Take derivatives of both sides of the equation.)
\vfill

\noindent \hrule

\bigskip
\centerline{\footnotesize \textsc{extra space for answers}}
\vfill
\end{document}