\documentclass[12pt]{article}

% Layout.
\usepackage[top=1.2in, bottom=0.75in, left=1in, right=1in, headheight=1.0in, headsep=0pt]{geometry}

% Fonts.
\usepackage{mathptmx}
\usepackage[scaled=0.86]{helvet}
\renewcommand{\emph}[1]{\textsf{\textbf{#1}}}

% TiKZ.
\usepackage{tikz, pgfplots}
\usetikzlibrary{calc}
\pgfplotsset{my style/.append style={axis x line=middle, axis y line=middle, xlabel={$x$}, ylabel={$y$}}}
\pgfplotsset{compat=1.16}

% Misc packages.
\usepackage{amsmath,amssymb,latexsym,bm,array,multicol,enumitem}
\usepackage{graphicx}
\usepackage{xcolor}

% Commands to set various header/footer components.
\makeatletter
\def\doctitle#1{\gdef\@doctitle{#1}}
\doctitle{Use {\tt\textbackslash doctitle\{MY LABEL\}}.}
\def\docdate#1{\gdef\@docdate{#1}}
\docdate{Use {\tt\textbackslash docdate\{MY DATE\}}.}
\def\doccourse#1{\gdef\@doccourse{#1}}
\let\@doccourse\@empty
\def\docscoring#1{\gdef\@docscoring{#1}}
\let\@docscoring\@empty
\def\docversion#1{\gdef\@docversion{#1}}
\let\@docversion\@empty
\makeatother

% Headers and footers layout.
\makeatletter
\usepackage{fancyhdr}
\pagestyle{fancy}
\fancyhf{} % Clears all headers/footers.
\lhead{\emph{\@doctitle\hfill\@docdate} \medskip
\ifnum \value{page} > 1\relax\else\\
\emph{Name: \rule{3.5in}{1pt}\hfill \@docscoring}
\fi}

\rfoot{\emph{\@docversion}}
\lfoot{\emph{\@doccourse}}
\cfoot{\emph{\thepage}}
\renewcommand{\headrulewidth}{0pt}%
\makeatother

% Paragraph spacing
\parindent 0pt
\parskip 6pt plus 1pt

% A problem is a section-like command. Use \problem{5} for a problem worth 5 points.
\newcounter{probcount}
\newcounter{subprobcount}
\setcounter{probcount}{0}

\newcommand{\problem}[1]{%
\par
\addvspace{4pt}%
\setcounter{subprobcount}{0}%
\stepcounter{probcount}%
\makebox[0pt][r]{\emph{\arabic{probcount}.}\hskip1ex}\emph{[#1 points]}\hskip1ex}

\newcommand{\thesubproblem}{\emph{\alph{subprobcount}.}}

% Subproblems are an enumerate-like environment with a consistent
% numbering scheme.  Use \begin{subproblems}\item...\item...\end{subproblems}
\newenvironment{subproblems}{%
\begin{enumerate}[itemindent=8pt,leftmargin=0pt]%
\setcounter{enumi}{\value{subprobcount}}%
\renewcommand{\labelenumi}{\emph{\textsl{\alph{enumi})}} \,}%
}%
{%
\setcounter{subprobcount}{\value{enumi}}%
\end{enumerate}%
}

% Blanks for answers in normal and math mode.
\newcommand{\blank}[1]{\rule{#1}{0.75pt}}
\newcommand{\mblank}[1]{\underline{\hspace{#1}}}
\def\emptybox(#1,#2){\framebox{\parbox[c][#2]{#1}{\rule{0pt}{0pt}}}}

% Misc.
\renewcommand{\d}{\displaystyle}
\newcommand{\ds}{\displaystyle}
\newcommand{\threeopts}{{\small \hspace{-6mm} $\begin{matrix} \text{\textsc{converges}} \\ \text{\textsc{absolutely}} \end{matrix}$ \qquad\qquad $\begin{matrix} \text{\textsc{converges}} \\ \text{\textsc{conditionally}} \end{matrix}$ \qquad\qquad \textsc{diverges}} \bigskip}


\newcommand{\ba}{\mathbf{a}}
\newcommand{\bb}{\mathbf{b}}
\newcommand{\bc}{\mathbf{c}}
\newcommand{\bi}{\mathbf{i}}
\newcommand{\bj}{\mathbf{j}}
\newcommand{\bk}{\mathbf{k}}
\newcommand{\bn}{\mathbf{n}}
\newcommand{\br}{\mathbf{r}}
\newcommand{\bu}{\mathbf{u}}
\newcommand{\bv}{\mathbf{v}}
\newcommand{\bw}{\mathbf{w}}

\newcommand{\bF}{\mathbf{F}}
\newcommand{\bG}{\mathbf{G}}
\newcommand{\bH}{\mathbf{H}}
\newcommand{\bT}{\mathbf{T}}

\newcommand{\grad}{\nabla}
\newcommand{\Div}{\nabla\cdot}

\newcommand{\ip}[2]{\mathrm{\left<#1,#2\right>}}
\newcommand{\vv}[2]{\mathrm{\left<#1,#2\right>}}
\newcommand{\vvv}[3]{\mathrm{\left<#1,#2,#3\right>}}


\doctitle{Math 253: Quiz 9}
\docdate{Thursday 13 April, 2023}
\doccourse{}
\docversion{}
\docscoring{\fbox{{\LARGE \strut}\hspace{0.8in} / 25}}

\begin{document}
30 minutes maximum.  No aids (book, calculator, etc.) are permitted.  Show all work and use proper notation for full credit.  Answers should be in reasonably-simplified form.  25 points possible.

% like 6.1 #8
\problem{4}  Sketch the vector field \,{\large  $\bF(x,y) = \bi - \bj$}.   Describe the vector field in a sentence.  (\textsl{Draw at least 6 vectors in your sketch.  Regarding the sentence, what is the length of the vectors, and their directions?})
\vfill

% like 6.1 #17
\problem{3}  Compute the gradient vector field of \,{\large $f(x,y,z) = xy - yz + z^3$.}
\vspace{3.0in}


\clearpage\newpage
% like 6.2 #53 but simpler
\problem{6}  Find the work done by the force field  \,{\large $\bF(x,y) = -2y\bi + 2x \bj + (x+y)\bk$} in moving an object {\large \strut} along the curve  \,{\large $\br(t) = \cos(t) \bi + \sin(t)\bj - 3 \bk$}, where $0 \le t \le \pi$.
\vfill

\problem{3}  Find $\bT(t)$, the unit tangent vector field, for the curve $\br(t) = \left<t,t^2,t^3\right>$.
\vspace{2.5in}


\clearpage\newpage
% like 6.2 #73, but easier
\problem{6}  Evaluate the line integral \,{\large $\displaystyle \int_C f(x,y)\,ds$}\, of the scalar function \,{\large $f(x,y)=x+y$}\, along the straight-line path connecting the origin to the point $(1,1)$.
\vfill


% like 6.1 #17
\problem{3}  Consider the vector fields $\bF = -y \bi + x\bj$, $\bG = x\bi + y \bj$, and $\bH = x \bi - y \bj$.  Match $\bF,\bG,\bH$ to their graphs below.  In particular, write the variable name below the matching graph.

\noindent\hspace{-10mm}
\mbox{\includegraphics[width=0.37\textwidth]{figs/F.jpg} \quad \includegraphics[width=0.37\textwidth]{figs/G.jpg} \quad \includegraphics[width=0.33\textwidth]{figs/H.jpg}}

\bigskip
\hspace{7mm} \underline{\phantom{SPACE FOR L}} \hspace{38mm} \underline{\phantom{SPACE FOR L}} \hspace{36mm} \underline{\phantom{SPACE FOR L}}

\bigskip\bigskip

\clearpage\newpage
\noindent \emph{Extra Credit. [1 point]} \,  I claim $\bF(x,y)=\left<e^x,y + e^x\right>$ is a conservative vector field.  Find a potential $f(x,y)$.
% is F=<...> a gradient? if so, find f so that grad  F  = f
\vfill

\noindent \hrule

\bigskip
\centerline{\footnotesize \textsc{extra space for answers}}
\vfill
\end{document}