\documentclass[12pt]{article}

% Layout.
\usepackage[top=1.2in, bottom=0.75in, left=1in, right=1in, headheight=1.0in, headsep=0pt]{geometry}

% Fonts.
\usepackage{mathptmx}
\usepackage[scaled=0.86]{helvet}
\renewcommand{\emph}[1]{\textsf{\textbf{#1}}}

% TiKZ.
\usepackage{tikz, pgfplots}
\usetikzlibrary{calc}
\pgfplotsset{my style/.append style={axis x line=middle, axis y line=middle, xlabel={$x$}, ylabel={$y$}}}
\pgfplotsset{compat=1.16}

% Misc packages.
\usepackage{amsmath,amssymb,latexsym,bm,array,multicol,enumitem}
\usepackage{graphicx}
\usepackage{xcolor}

% Commands to set various header/footer components.
\makeatletter
\def\doctitle#1{\gdef\@doctitle{#1}}
\doctitle{Use {\tt\textbackslash doctitle\{MY LABEL\}}.}
\def\docdate#1{\gdef\@docdate{#1}}
\docdate{Use {\tt\textbackslash docdate\{MY DATE\}}.}
\def\doccourse#1{\gdef\@doccourse{#1}}
\let\@doccourse\@empty
\def\docscoring#1{\gdef\@docscoring{#1}}
\let\@docscoring\@empty
\def\docversion#1{\gdef\@docversion{#1}}
\let\@docversion\@empty
\makeatother

% Headers and footers layout.
\makeatletter
\usepackage{fancyhdr}
\pagestyle{fancy}
\fancyhf{} % Clears all headers/footers.
\lhead{\emph{\@doctitle\hfill\@docdate} \medskip
\ifnum \value{page} > 1\relax\else\\
\emph{Name: \rule{3.5in}{1pt}\hfill \@docscoring}
\fi}

\rfoot{\emph{\@docversion}}
\lfoot{\emph{\@doccourse}}
\cfoot{\emph{\thepage}}
\renewcommand{\headrulewidth}{0pt}%
\makeatother

% Paragraph spacing
\parindent 0pt
\parskip 6pt plus 1pt

% A problem is a section-like command. Use \problem{5} for a problem worth 5 points.
\newcounter{probcount}
\newcounter{subprobcount}
\setcounter{probcount}{0}

\newcommand{\problem}[1]{%
\par
\addvspace{4pt}%
\setcounter{subprobcount}{0}%
\stepcounter{probcount}%
\makebox[0pt][r]{\emph{\arabic{probcount}.}\hskip1ex}\emph{[#1 points]}\hskip1ex}

\newcommand{\thesubproblem}{\emph{\alph{subprobcount}.}}

% Subproblems are an enumerate-like environment with a consistent
% numbering scheme.  Use \begin{subproblems}\item...\item...\end{subproblems}
\newenvironment{subproblems}{%
\begin{enumerate}[itemindent=8pt,leftmargin=0pt]%
\setcounter{enumi}{\value{subprobcount}}%
\renewcommand{\labelenumi}{\emph{\textsl{\alph{enumi})}} \,}%
}%
{%
\setcounter{subprobcount}{\value{enumi}}%
\end{enumerate}%
}

% Blanks for answers in normal and math mode.
\newcommand{\blank}[1]{\rule{#1}{0.75pt}}
\newcommand{\mblank}[1]{\underline{\hspace{#1}}}
\def\emptybox(#1,#2){\framebox{\parbox[c][#2]{#1}{\rule{0pt}{0pt}}}}

% Misc.
\renewcommand{\d}{\displaystyle}
\newcommand{\ds}{\displaystyle}
\newcommand{\threeopts}{{\small \hspace{-6mm} $\begin{matrix} \text{\textsc{converges}} \\ \text{\textsc{absolutely}} \end{matrix}$ \qquad\qquad $\begin{matrix} \text{\textsc{converges}} \\ \text{\textsc{conditionally}} \end{matrix}$ \qquad\qquad \textsc{diverges}} \bigskip}


\newcommand{\ba}{\mathbf{a}}
\newcommand{\bb}{\mathbf{b}}
\newcommand{\bc}{\mathbf{c}}
\newcommand{\bi}{\mathbf{i}}
\newcommand{\bj}{\mathbf{j}}
\newcommand{\bk}{\mathbf{k}}
\newcommand{\bn}{\mathbf{n}}
\newcommand{\br}{\mathbf{r}}
\newcommand{\bu}{\mathbf{u}}
\newcommand{\bv}{\mathbf{v}}
\newcommand{\bw}{\mathbf{w}}

\newcommand{\bT}{\mathbf{T}}

\newcommand{\grad}{\nabla}
\newcommand{\Div}{\nabla\cdot}

\newcommand{\ip}[2]{\mathrm{\left<#1,#2\right>}}
\newcommand{\vv}[2]{\mathrm{\left<#1,#2\right>}}
\newcommand{\vvv}[3]{\mathrm{\left<#1,#2,#3\right>}}


\doctitle{Math 253: Quiz 8}
\docdate{Thursday 30 March, 2023}
\doccourse{}
\docversion{}
\docscoring{\fbox{{\LARGE \strut}\hspace{0.8in} / 25}}

\begin{document}
30 minutes maximum.  No aids (book, calculator, etc.) are permitted.  Show all work and use proper notation for full credit.  Answers should be in reasonably-simplified form.  25 points possible.

% 5.3 #154
\problem{5}  Find the area of one leaf of the rose $r=\sin (2\theta)$, which is shown in the figure.

\bigskip \hfill \includegraphics[width=0.35\textwidth]{figs/fourleaf.jpg}
\vfill

% like 5.3 #151, but without doing integral
\problem{5}  Convert the integral to polar coordinates.  There is no need to evaluate the integral!  (\emph{Hint.} Sketch the region of integration, which tells you the limits on the $r,\theta$ integrals.)

\vspace{-7mm}

$$\int\displaylimits_0^4 \int\displaylimits_{\,-\sqrt{16-x^2}}^{\,+\sqrt{16-x^2}} \arctan(x^2+y^2)\,dy\,dx = \hspace{4.0in}$$
\vfill


\clearpage\newpage
% 5.4 #190
\problem{5}  Using mathematically-correct steps, show that:
    $$\int_a^b \int_c^d \int_e^f F'(x) G'(y) H'(z)\,dz\,dy\,dx = \left[F(b)-F(a)\right]\, \left[G(d)-G(c)\right]\, \left[H(f)-H(e)\right]$$
(\emph{Hint.}  Start on the left.  What terms can be moved out of the inner integrals?  What do you know about the integral of a derivative?)
\vfill

% 5.4 # 182, but simpler
\problem{5}  Assume $\ds B=\left\{(x,y,z)\,\big|\,1 \le x \le 2, \, 0 \le y \le 2, \, 1 \le z \le 3 \right\}$.  Evaluate the triple integral:
  $$\iiint\displaylimits_B \,xy\,dV = \hspace{5.0in}$$
\vfill


\clearpage\newpage
% from 5.4 #232
\problem{5}  A solid object is shown.  It is the set in the first octant which bounded by $z=1-x^2$ and the plane $y=5$.  Supposing its density is $\rho(x,y,z) = 1 + x + y$, \textbf{completely set up} a triple integral to find its total mass.

\bigskip \hfill \includegraphics[width=0.45\textwidth]{figs/halfhump.jpg}
\vfill

\noindent \hrule

\bigskip
\centerline{\footnotesize \textsc{extra space for answers}}
\vfill

\clearpage\newpage
\noindent \emph{Extra Credit. [1 point]} \, Compute and fully simplify the integral in problem \textbf{5}.
\vfill
\end{document}