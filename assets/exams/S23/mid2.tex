\documentclass[11pt]{amsart}
%\pagestyle{empty} 
\setlength{\topmargin}{-0.5in} % usually -0.25in
\addtolength{\textheight}{1.2in} % usually 1.25in
\addtolength{\oddsidemargin}{-0.95in}
\addtolength{\evensidemargin}{-0.95in}
\addtolength{\textwidth}{1.9in} %\setlength{\parindent}{0pt}

\newcommand{\normalspacing}{\renewcommand{\baselinestretch}{1.1}\tiny\normalsize}
\normalspacing

% macros
\usepackage{amssymb,xspace,alltt,verbatim}
\usepackage[final]{graphicx}
\usepackage[pdftex,colorlinks=true]{hyperref}
\usepackage{fancyvrb}
\usepackage{tikz}

\newtheorem*{lem*}{Lemma}

\newcommand{\ba}{\mathbf{a}}
\newcommand{\bb}{\mathbf{b}}
\newcommand{\bc}{\mathbf{c}}
\newcommand{\bbf}{\mathbf{f}}
\newcommand{\bi}{\mathbf{i}}
\newcommand{\bj}{\mathbf{j}}
\newcommand{\bk}{\mathbf{k}}
\newcommand{\br}{\mathbf{r}}
\newcommand{\bs}{\mathbf{s}}
\newcommand{\bu}{\mathbf{u}}
\newcommand{\bv}{\mathbf{v}}
\newcommand{\bw}{\mathbf{w}}
\newcommand{\bx}{\mathbf{x}}
\newcommand{\by}{\mathbf{y}}

\newcommand{\bT}{\mathbf{T}}

\newcommand{\CC}{{\mathbb{C}}}
\newcommand{\RR}{{\mathbb{R}}}
\newcommand{\eps}{\epsilon}
\newcommand{\ZZ}{{\mathbb{Z}}}
\newcommand{\NN}{{\mathbb{N}}}
\newcommand{\ip}[2]{\mathrm{\left<#1,#2\right>}}

\renewcommand{\Re}{\operatorname{Re}}
\renewcommand{\Im}{\operatorname{Im}}
\newcommand{\Log}{\operatorname{Log}}
\newcommand{\grad}{\nabla}

\newcommand{\ds}{\displaystyle}

\newcommand{\Matlab}{\textsc{Matlab}\xspace}

\newcommand{\prob}[1]{\bigskip\noindent\textbf{#1.} }
\newcommand{\pts}[1]{(\emph{#1 pts})}

\newcommand{\probpts}[2]{\prob{#1} \pts{#2} \quad}
\newcommand{\ppartpts}[2]{\textbf{(#1)} \pts{#2} \quad}
\newcommand{\epartpts}[2]{\medskip\noindent \textbf{(#1)} \pts{#2} \quad}


\begin{document}
\hfill \Large Name:\underline{\phantom{Ed Bueler really really long long long name}}
\medskip

\scriptsize \noindent Math 253 Calculus III (Bueler) \hfill Thursday, 6 April 2023
\medskip

\LARGE\centerline{\textbf{Midterm Exam 2}}

\smallskip
\begin{quote}
\large
\textbf{No book, notes, electronics, calculator, or internet access.  100 points possible.  70 minutes maximum.}
\end{quote}

\normalsize
\medskip

\thispagestyle{empty}

%%% BUILD FROM
% quizzes 5,6,7,8 AND sections
% 4.4 tangent planes and linear approx
% 4.5 chain rules
% 4.6 gradient and directional derivs
% 4.7 optimization
% 5.1 double integrals: rectangular regions
% 5.2 double integrals: general regions
% 5.3 double integrals in polar
% 5.4 triple integrals
% 2.7 cyl. and spher. coords
% 5.5 triple integrals in cyl. and spher. coords


% 4.4 #173 (on homework)
\probpts{1}{10}  Find an equation of the tangent plane of the surface \,{\large $z = \ln(10 x^2 + 2 y^2 + 1)$}\, at the point \,{\large $P(0,0,0)$}.
\vfill

\prob{2}  \ppartpts{a}{5}   Suppose \,{\large $f(u,v)$}\, is a function of two variables, and that, in turn, \,{\large $u=u(r,s)$}\, and \,{\large $v=v(r,s)$}.  Write down the chain rule which computes \,{\large $\frac{\partial f}{\partial s}$}:

\bigskip
    $$\frac{\partial f}{\partial s} = \hspace{6.0in}$$
\vspace{0.5in}

% like 4.5 #245 (on homework), but simpler
\epartpts{b}{5}  Specifically suppose \,{\large $f(u,v)=uv$},\, {\large $u(r,s)=r \cos s$},\, and \, {\large $v(r,s)=r \sin s$}.  Compute the following partial derivative, and express your answer as a simplified expression in variables $r,s$.

\bigskip
    $$\frac{\partial f}{\partial s} = \hspace{6.0in}$$
\vspace{1.0in}

\clearpage\newpage
% easier than 4.7 #324 (on homework)
\prob{3}  Consider the function \,{\large $f(x,y) = x^3 + y^3 - 12 x - 15 y + 7$}.

\epartpts{a}{5}  Compute the gradient:

\bigskip
    $$\grad f(x,y) = \hspace{6.0in}$$
\vspace{0.25in}

\epartpts{b}{5}  Find all of the critical points.  Write each one as a pair $(x,y)$.
\vfill

\epartpts{c}{5}  Use the second derivative test to classify all of the critical points, as local maximum, local minimum, or saddle point.
\vfill

\epartpts{d}{3}  Consider the square \,{\large $D = [-10,10]\times [-10,10]$}.  Does the absolute maximum of \,{\large $f(x,y)$}\, over \,{\large $D$}\, occur at one of the critical points found in part \textbf{(b)}?  State the answer \textbf{yes} or \textbf{no}, and explain in one sentence.

\medskip \noindent
(\emph{Hint.}  You do not need to find the absolute maximum!  But consider values of $f(x,y)$ when answering.)
\vspace{1.0in}


\clearpage\newpage
% easier than 5.2 #77
\prob{4}  \ppartpts{a}{5}  Sketch the region

\medskip
\noindent {\large $D = \Big\{(x,y)\,\big|\,-1\le y \le 0, \, 0 \le x \le 1-y^2\Big\}$}.
\vfill

\epartpts{b}{10}  Compute and simplify the double integral over the region $D$ in part \textbf{(a)}:

\bigskip
$$\iint_D xy\,dA = \hspace{5.0in}$$
\vspace{2.5in}

% Quiz 8 question verbatim
\probpts{5}{10}  \underline{Set up, and then evaluate,} a double integral for the area of one leaf of the rose $r=\sin (2\theta)$, as shown in the figure.

\hfill \includegraphics[width=0.32\textwidth]{figs/fourleaf.jpg}
\vspace{1.0in}



\clearpage\newpage
% 5.4 #183, from homework, simplified
\probpts{6}{10}  Define \,{\large $B = \Big\{(x,y,z)\,\big|\, 0\le x\le 1, \, 0 \le y \le \pi, \, -1 \le z \le 1\Big\}$}, a rectangular solid box.  Evaluate the triple integral: 

\bigskip
$$\iiint_B x \sin y\,dV = \hspace{5.0in}$$
\vfill

\probpts{7}{10}  The region \,{\large $E = \Big\{(x,y,z)\,\big|\, x^2 + y^2 \le 1, \, x \ge 0, \, x \ge y, \, -1 \le z \le 1\Big\}$} is shown below.  Use \textbf{cylindrical coordinates} to evaluate the triple integral \,{\large $\iiint_E f(x,y,z) \,dV$} \, if $f(x,y,z)=xy$.

\bigskip \hfill \includegraphics[width=0.35\textwidth]{figs/cheesewedge.jpg}
\vfill


\clearpage\newpage
\probpts{8}{10}  Find the average value of the function \,{\large $f(x,y,z)=x^2+y^2+z^2$} \, over the unit sphere  \,{\large $S = \Big\{(x,y,z)\,\big|\, x^2 + y^2 + z^2 \le 1\Big\}$}.

\medskip \noindent
(\emph{Hints.}  What coordinates would make this easiest?  Yes, you \emph{may} use the fact that the volume of the unit sphere is $4\pi/3$; there is no need to justify it.)
\vspace{4.5in}

\probpts{9}{7}  Use \textbf{spherical coordinates} to \underline{fully set up} a triple integral for the volume which is outside the cone \, {\large $z^2=x^2+y^2$} \, but inside the unit sphere \, {\large $x^2+y^2 + z^2 =1$}.  \underline{Do not evaluate the integral.}

\medskip \noindent
(\emph{Hint.}  Start by drawing a decent sketch!)
\vfill

\clearpage\newpage
\probpts{Extra Credit}{3}  The equation \, {\large $x^2+y^2=9$} \, is a cylinder.  Convert this equation to \textbf{spherical coordinates}, and write your simplified answer in the form {\large $\rho=f(\varphi,\theta)$}.
\vfill

\bigskip
\noindent \hrule
\begin{center}
\small
\bigskip
\textsc{blank space}
\end{center}
\vfill

\end{document}
