\documentclass[11pt]{amsart}
%\pagestyle{empty} 
\setlength{\topmargin}{-0.5in} % usually -0.25in
\addtolength{\textheight}{1.2in} % usually 1.25in
\addtolength{\oddsidemargin}{-0.95in}
\addtolength{\evensidemargin}{-0.95in}
\addtolength{\textwidth}{1.9in} %\setlength{\parindent}{0pt}

\newcommand{\normalspacing}{\renewcommand{\baselinestretch}{1.1}\tiny\normalsize}
\normalspacing

% macros
\usepackage{amssymb,xspace,alltt,verbatim}
\usepackage[final]{graphicx}
\usepackage[pdftex,colorlinks=true]{hyperref}
\usepackage{fancyvrb}
\usepackage{tikz}

\newtheorem*{lem*}{Lemma}

\newcommand{\ba}{\mathbf{a}}
\newcommand{\bb}{\mathbf{b}}
\newcommand{\bc}{\mathbf{c}}
\newcommand{\bbf}{\mathbf{f}}
\newcommand{\bi}{\mathbf{i}}
\newcommand{\bj}{\mathbf{j}}
\newcommand{\bk}{\mathbf{k}}
\newcommand{\br}{\mathbf{r}}
\newcommand{\bs}{\mathbf{s}}
\newcommand{\bu}{\mathbf{u}}
\newcommand{\bv}{\mathbf{v}}
\newcommand{\bw}{\mathbf{w}}
\newcommand{\bx}{\mathbf{x}}
\newcommand{\by}{\mathbf{y}}

\newcommand{\bT}{\mathbf{T}}

\newcommand{\CC}{{\mathbb{C}}}
\newcommand{\RR}{{\mathbb{R}}}
\newcommand{\eps}{\epsilon}
\newcommand{\ZZ}{{\mathbb{Z}}}
\newcommand{\NN}{{\mathbb{N}}}
\newcommand{\ip}[2]{\mathrm{\left<#1,#2\right>}}

\renewcommand{\Re}{\operatorname{Re}}
\renewcommand{\Im}{\operatorname{Im}}
\newcommand{\Log}{\operatorname{Log}}
\newcommand{\grad}{\nabla}

\newcommand{\ds}{\displaystyle}

\newcommand{\Matlab}{\textsc{Matlab}\xspace}

\newcommand{\prob}[1]{\bigskip\noindent\textbf{#1.} }
\newcommand{\pts}[1]{(\emph{#1 pts})}

\newcommand{\probpts}[2]{\prob{#1} \pts{#2} \quad}
\newcommand{\ppartpts}[2]{\textbf{(#1)} \pts{#2} \quad}
\newcommand{\epartpts}[2]{\medskip\noindent \textbf{(#1)} \pts{#2} \quad}


\begin{document}
\hfill \Large Name:\underline{\phantom{Ed Bueler really really long long long name}}
\medskip

\scriptsize \noindent Math 253 Calculus III (Bueler) \hfill Thursday, 23 February 2023
\medskip

\LARGE\centerline{\textbf{Midterm Exam 1}}

\smallskip
\begin{quote}
\large
\textbf{No book, notes, electronics, calculator, or internet access.  100 points possible.  70 minutes maximum.}
\end{quote}

\normalsize
\medskip

\thispagestyle{empty}

% Q1 1bcd, plus
\prob{1}  Suppose we have three vectors, \, $\ba = \bi - \bj$, \, $\bb = \bj + 3\bk$,\, $\bc=-\bi+2\bj-4\bk$.  Compute the following quantities which are either scalars or vectors.  You can write the vectors using either component notation or standard unit vector notation.

\epartpts{a}{5} $(\ba\cdot \bb) \,\bc=$
\vfill

\epartpts{b}{5} a unit vector in the direction of $\bc$: \quad $\bu=$
\vfill

\epartpts{c}{5} the angle between vectors $\ba$ and $\bb$: \quad $\theta=$
\vfill

\clearpage\newpage
% Q2 #3
\prob{2} \ppartpts{a}{10}  Find a general equation of the plane through the three points $P(3,-1,2)$, $Q(1,0,1)$, and $R(0,-1,1)$.  Express your answer in the form $ax+by+cz+d=0$. % like 2.5 # 281
\vfill

% Q2 #5
\epartpts{b}{5}  Consider the same three points as in part \textbf{(a)}.  Find the area of the triangle $PQR$.  % like 2.4 # 211(b)
\vspace{3.0in}

\clearpage\newpage
% like 4.3 #121
\prob{3}  Suppose $z = \ln(xy + y^4)$.  Compute the following partial derivatives.  There is no need to simplify your answers.

\medskip
\epartpts{a}{5}  \quad $\ds \frac{\partial z}{\partial y}=$
\vfill

\epartpts{b}{5}  \quad $\ds \frac{\partial^2 z}{\partial x \partial y}=$
\vfill

\probpts{4}{5}  \underline{Find} and \underline{sketch} (shade in) the domain of the function $f(x,y) = \sqrt{x^2 + y^2 - 9}$.  Fill in the set notation below.

\bigskip\medskip
$\ds (\text{domain of } f) = \left\{(x,y)\,\Big|\, \phantom{kldsfjklajsdkjklj sdfjasdfl} \right\}$
\vfill


\clearpage\newpage
% Q2 #4
\probpts{5}{5}  Find a vector equation of the line passing through the two points $P(4,0,5)$ and $Q(2,3,1)$.  % 2.5 # 244
\vfill

% Q3 #3b
\probpts{6}{10}  Suppose that a moving particle has position function {\large\, $\br(t) = \left<e^{-t},t,t^2\right>$}.  Calculate the tangent line to the curve {\large \,$\br(t)$\,} at {\large $t=1$}.  (\emph{Hint}.  The answer can be vector-valued or parametric.)
\vfill


\clearpage\newpage
% Q4 #2
\probpts{7}{5} Compute the arc length of the helix $\br(t) = \left<\cos t,\sin t,t\right>$ from $t=-2$ to $t=0$.
\vfill

\probpts{8}{5}  Write the curve (graph) $y=f(x)$ as a vector-valued curve $\br(t)$.
\vspace{2.5in}

\probpts{9}{5}  Compute the limit:

$\ds \lim_{(x,y)\to (1,1)} \frac{xy - x}{y^2 - 1} =$
\vspace{2.5in}


\clearpage\newpage
% like 3.3 #130
\prob{10}  \ppartpts{a}{5}  Find $\bT(t)$, the unit tangent vector, for the circle $\br(t) = \left<3\cos t,3\sin t\right>$.
\vfill

\epartpts{b}{5}  Compute the curvature of the curve $\br(t)$ in part \textbf{(a)}, at the point $t=0$.
\vfill

% example 4.9 in section 4.2
\probpts{Extra Credit}{3}  Show that the following limit does not exist:

\bigskip
$\ds \lim_{(x,y) \to (0,0)} \frac{2 x y}{3x^2 + y^2}$
\vspace{2.0in}

\clearpage\newpage
\probpts{11}{5}  Are the two planes $x - 2 y + 3 z = 5$ and $- 2 x + 4 y - 6 z = 0$ parallel?  If so, explain why.  If not, find the angle between the planes.
\vfill

\probpts{12}{10}  Find the distance from the point $P(0,0,1)$ to the plane $x+2y+3z=4$.  (\emph{Hint.}  To start, draw a sketch and find a concrete point which is in the plane.  Now, what vectors do you know?)
\vfill

\clearpage\newpage
\noindent You may find the following curvature formulas useful.  However, there are many other formulas, not listed here, which you should have in memory.

\begin{align*}
\kappa(s) &= \left\|\frac{d\bT}{ds}\right\| \\
\kappa(t) &= \frac{\|\bT'(t)\|}{\|\br'(t)\|} &&\text{for curves $\br(t)$} \\
\kappa(x) &= \frac{|f''(x)|}{\left(1+f'(x)^2\right)^{3/2}} &&\text{for curves $y=f(x)$}
\end{align*}

\bigskip
\noindent \hrule
\begin{center}
\small
\bigskip
\textsc{blank space}
\end{center}
\vfill

\end{document}
